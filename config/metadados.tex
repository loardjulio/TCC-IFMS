%% %%%%%%%%%%%%%%%%%%%%%%%%%%%%%%%%%%%%%%%%%%%%%%%% %%
%% Metadados do trabalho
%% Informações de dados para CAPA e FOLHA DE ROSTO
%% AVISO: Todos esses dados serão automaticamente convertidos para caixa alta onde necessário
%% %%%%%%%%%%%%%%%%%%%%%%%%%%%%%%%%%%%%%%%%%%%%%%%% %%

\titulo{Sistema de gerenciamento de estoque para centros de jardinagem }

\autor{Julio César Alves da Silva}
\local{Naviraí}
\data{2018}

%% "M\textsuperscript{e}." = Abreviação oficial para "Mestre"
\orientador{Prof. M\textsuperscript{e}. Jean Carlo Wai Keung Ma}
%\coorientador{Clara Castro Cardoso}

\instituicao{%
  INSTITUTO FEDERAL DE EDUCAÇÃO, CIÊNCIA E TECNOLOGIA DE MATO GROSSO DO SUL
  \par
  CAMPUS NAVIRAÍ
  \par
  TECNOLOGIA EM ANÁLISE E DESENVOLVIMENTO DE SISTEMAS}
  
%% Tipo de Trabalho
%% - Monografia
%% - Tese (Mestrado)
%% - Tese (Doutorado)
%% - Relatório técnico
\tipotrabalho{Monografia}

% O preambulo deve conter o tipo do trabalho, o objetivo, 
% o nome da instituição e a área de concentração 
\preambulo{Trabalho de conclusão de curso apresentado à Banca Examinadora como requisito para aprovação na disciplina de Projeto Integrador I do Curso Superior de Tecnologia em Análise e Desenvolvimento de Sistemas do Instituto Federal de Educação, Ciência e Tecnologia de Mato Grosso do Sul.}

%% Primeiro membro da banca examinadora
\membroum{Prof. M\textsuperscript{e}.  Alisson Gaspar Chiquitto                }
% {Instituto Federal de Educação, Ciência e Tecnologia de Mato Grosso do Sul - IFMS}

%% Segundo membro da banca examinadora
\membrodois{Prof. M\textsuperscript{e}.            }

%% Terceiro membro da banca examinadora - Se houver
% \membrotres{Prof. Dr. Ney Inga de Oliveira Nome do Membro 3}

%% Data da apresentação do trabalho
%% Se não souber a data da apresentação, utilize \underline{\hspace{3.5cm}}
%% Isso cria um sublinhado de 3.5cm, onde você pode escrever a data depois!
%\dataapresentacao{ de ______ de 2017}
\dataapresentacao{\underline{\hspace{3.5cm}}}