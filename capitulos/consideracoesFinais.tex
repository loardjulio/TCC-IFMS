\chapter{Considerações Finais}

Até aqui, apresentamos uma solução para as demandas encontradas em um centro de jardinagem, onde o objetivo maior foi colaborar com a eficiência no processo de geração de orçamentos, possibilitando aos clientes realizarem a análise dos orçamentos gerados. Além disso, fornecemos uma ferramenta que permite gerenciar clientes, fornecedores e produtos do estoque de maneira simples e direta, de tal maneira, que estes processos requerem pouco tempo e esforço para os operadores do sistema. 
Durante o processo de desenvolvimento, foram utilizadas várias técnicas aprendidas em sala de aula, técnicas estas aplicadas durante a concepção do sistema até a fase implementação do mesmo.
%Portanto, o sistema se apresenta com grande potencial de solução para os problemas detectados.

A concepção das necessidades do sistema foram identificados em atividades de disciplinas anteriores. Desde o início, já estava ciente das possíveis dificuldades que encontraria ao longo do processo de desenvolvimento. A implementação de algumas funcionalidades se mostraram um pouco conturbadas, como no caso do salvamento e da geração de orçamentos, junção entre as tabelas e mapeamento das entidades. Tais atividades pareciam simples na teoria e se mostraram problemáticas no decorrer do desenvolvimento. Entretanto, essas dificuldades foram sendo superadas ao longo do projeto e dava mais energia para continuar o desenvolvimento. Além disso, pude descobrir novas ferramentas e tecnologias que poderiam ser aplicadas neste projeto, como no caso do \textit{Maven} e \textit{Hibernate Validator}, Assim, devido ao estágio de desenvolvimento em que se encontra, acabamos impossibilitados de fazer sua implantação de imediato, abrindo possibilidades para sugestões de trabalhos futuros. Sendo assim, apresentamos abaixo alguns pontos que podem ser realizados em trabalhos futuros:

\begin{itemize}
    \item Utilização do \textit{Maven} e \textit{Hibernate Validator}.
    \item Aplicar o padrão de projeto em 3 camadas (\textit{Model View Controller}).
    \item Possibilidade de uso do sistema em atividades de outro ramo comercial.
    \item Comparativo de preços de produtos entre fornecedores.
    \item Gerenciamento de projeto de paisagismo.
\end{itemize}

Após a conclusão do sistema, antes da implantação, deve ser submetido as etapas de validação e verificação, cujo objetivo é garantir que o sistema atenda às expectativas do cliente e cumpra todos os requisitos especificados. Após estas etapas, ocorrerá a entrega do sistema, onde o mesmo passará a ser utilizado no cotidiano no centro de jardinagem.
