\chapter{Introdução}
Muitas pessoas ainda acreditam que o paisagismo é apenas uma elaboração de praças e jardins. O que muitos não sabem é que trata-se de uma técnica cada vez mais apurada para substituir áreas destruídas pelo homem. A missão do paisagista é reconstruir essas áreas, planejando e gerenciando o ambiente físico com elementos da natureza. Um centro de jardinagem é um empreendimento varejista voltado a produtos relacionados a paisagismo e jardinagem. Possui um estoque variado com plantas, vasos, fertilizantes e acessórios, por isso, é o local buscado por pessoas comuns e profissionais de paisagismo para adquirir estes produtos. %aliando-se de recursos artesanais à percepção estética combinando cores e formatos para melhorar a convivência do ser humano com a natureza.

\section{Justificativa e Motivação}

    %Este projeto irá possibilitar o desenvolvimento de um sistema que será desenvolvido após amplas observações em um estabelecimento comercial da área de estudo, denominado Café com Flores Floricultura e Paisagismo. 
    Este projeto irá possibilitar o desenvolvimento de um sistema após amplas observações em um estabelecimento comercial atuante na área de jardinagem e na elaboração de projetos de paisagismo. Durante estas observações foram constatadas deficiências operacionais que podem ser solucionadas ou minimizadas com o apoio de um sistema informatizado.
    
     %Após a elaboração de um projeto de paisagismo deve-se elaborar um orçamento para os clientes com base nas plantas utilizadas. sendo este um trabalho que demanda certo tempo e esforço, pois os usuários consultam preços e disponibilidades dos produtos e serviços em estoque próprio  e com cada fornecedor de forma individual, tornando o trabalho repetitivo.
 Após a elaboração de um projeto de paisagismo deve-se elaborar um orçamento para os clientes com base nas plantas utilizadas. A elaboração de um orçamento demanda certo tempo e esforço, pois os usuários consultam os preços e as disponibilidades dos produtos no próprio centro de jardinagem ou com fornecedores de forma individual, tornando o trabalho repetitivo. 
    O sistema ajudará no processo de elaboração de orçamentos utilizando como base os produtos cadastrados no estoque ou cadastrados com base em informações de fornecedores. 


\section{Objetivo geral}

    O foco deste trabalho é apresentar as etapas de desenvolvimento de um sistema informatizado que atue no gerenciamento de estoque  e na geração de orçamentos, tendo em vista a necessidade das empresas em obter informações sobre preços e produtos na área de jardinagem e paisagismo.
 O objetivo é apresentar uma proposta que facilite o trabalho dos usuários no controle de estoque e na geração de orçamentos.



\section{Objetivo específico}

%O sistema busca superar as dificuldades encontradas no controle de estoque e na elaboração de orçamentos rápidos e confiáveis utilizando informações sobre o estoque existente, além de dados de fornecedores e parceiros cadastrados.

%O sistema deve emitir relatórios detalhados que forneça ao comerciante a informação sobre o fluxo dos produtos, a fim de se elaborar estratégias de mercado. E possibilitar a consulta e comparação de preços e produtos de fornecedores. Além disso, auxiliar no controle de atividades cotidianas inerentes ao processo de desenvolvimento de projetos de paisagismo. Ao final, espera-se uma otimização do trabalho suprindo as necessidades dos usuários.

O sistema deverá atingir os seguintes objetivos:

\begin{itemize}
   \item Controle de estoque
   \item Geração de orçamentos rápidos
   \item Cadastro de produtos de fornecedores
   \item Relatórios detalhados
   
 \end{itemize}




\section{Organização do trabalho}
%No capítulo 1 é apresentado uma introdução e contextualização geral dos objetivos deste trabalho e dos problemas encontrados. 
    %Este trabalho está dividido e organizado nos seguintes capítulos.
    %O capítulo 2 apresenta um referencial teórico onde são mostrados alguns sistemas concorrentes com suas funções, vantagens e desvantagens.
    %O capítulo 3 apresenta as metodologias e ferramentas adotadas para o desenvolvimento.
    %O capítulo 4 descreve as etapas de desenvolvimento até o momento, fornecendo uma visão geral do trabalho, seus requisitos funcionais e não-funcionais e os diagramas desenvolvidos através da Linguagem de Modelagem Unificada (UML).
    %Por fim, o capítulo 5 trata-se das considerações até o momento, descrevendo as dificuldades encontradas e os trabalhos futuros.
    %Por fim, o capítulo 5 trata-se das considerações finais descrevendo as dificuldades encontradas e os trabalhos futuros.
    
     Este trabalho está dividido e organizado nos seguintes capítulos.

\begin{itemize}
   \item Capítulo 2 apresenta um referencial teórico onde são mostrados alguns sistemas concorrentes com suas funções, vantagens e desvantagens.
   \item Capítulo 3 apresenta as metodologias e ferramentas adotadas para o desenvolvimento.
   \item Capítulo 4 descreve as etapas de desenvolvimento, fornecendo uma visão geral do trabalho, seus requisitos funcionais e não-funcionais, diagramas desenvolvidos através da Linguagem de Modelagem Unificada (UML) e Telas do sistema.
   \item Capítulo 5 trata-se das considerações finais, descrevendo as dificuldades encontradas e os trabalhos futuros.
   
 \end{itemize}






